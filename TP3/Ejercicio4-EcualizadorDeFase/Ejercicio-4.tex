
\subsection{Ecualizador de Fase}

En esta sección del informe se procederá a analizar un ecualizador 
de fase y control de tonos, cuyo
circuito se presenta a continuación en la figura [\ref{circuito_ecualizador_de_fase}].


\begin{figure}[H]
	\centering
	\includegraphics[width=0.45\textwidth]{../Ejercicio4-EcualizadorDeFase/Informe/Ecualizador_de_Fase.png}
	\caption{Circuito Ecualizador de Fase}
	\label{circuito_ecualizador_de_fase}
\end{figure}


\subsection{Análisis matemático}
A continuación, se presenta el desarrollo matemático pertinente para 
obtener la función transferencia del circuito planteado por la cátedra. \par 


Para analizar el circuito propuesto, se opto por reemplazar el potenciómetro $R_2$ 
por dos resistencias las cuales llamaremos 
$R_{21}$ y $R_{22}$, relacionadas por un coeficiente $L$. 
De esta forma será más fácil poder resolver el circuito propuesto, entonces definimos:

\vspace{1mm}
\begin{align}
		R_{21} &= R_2  L  \label{expresion_1} \\ 
		R_{22} &= R_2  (1 - L) \label{expresion_2}
\end{align}
\vspace{2mm}

Las relaciones entre ambas resistencias se plantean a continuación:

\vspace{2mm}
\begin{align}
	R_{21} + R_{22} &= R_2  L + R_2  (1 - L) = R_2 \label{suma_pote} \\ 	
	R_{21}  R_{22} &= R_2  L  R_2  (1 - L) = R_2^2(L-L^2)
	\label{multiplicacion_pote}
\end{align}
\vspace{2mm}

Usaremos las ecuaciones planteadas en [\ref{suma_pote}] y [\ref{multiplicacion_pote}] para 
simplificar las ecuaciones de la transferencia más adelante.


\begin{figure}[H]
	\centering
	\includegraphics[width=0.45\textwidth]{../Ejercicio4-EcualizadorDeFase/Informe/EcSinPot.png}
	\caption{Modelo matemático}
\end{figure}
Para simplificar el circuito se procedió a aplicar las transformaciones 
de Kenelly en dos oportunidades. Primero, se procedió a una conversión 
triángulo-estrella para luego utilizar su contraparte estrella-triángulo. 
Este procedimiento  se muestra en las figuras [\ref{1reemplazo}]
  y [\ref{2reemplazo}]. \par 
Para el primer reemplazo se usaron las siguientes ecuaciones:

\begin{align*}
		Z_{AB} &= \frac{1}{sC_1} \\ 
		Z_{BC} &= R_{22} \\ 
		Z_{CA} &= R_{21}
\end{align*}
\vspace{2mm}

Considerando,

\begin{align*}
	Z_{eq} &= Z_{AB} + Z_{BC} + Z_{CA}
\end{align*}
\vspace{2mm}

Se procede a realizar la primera transformación de Kenelly.
 Obteniéndose: \par 

\begin{align}
	Z_{A} &= \frac{Z_{AB}+Z_{CA}}{Z_{eq}} \\
	Z_{B} &= \frac{Z_{AB}+Z_{BC}}{Z_{eq}}  \\
	Z_{C} &= \frac{Z_{BC}+Z_{CA}}{Z_{eq}} 
\end{align}
\vspace{2mm}

\begin{figure}[H]

	\centering
	\includegraphics[width=0.9\textwidth]{../Ejercicio4-EcualizadorDeFase/Informe/1cambioEstrella.png}
	\caption{1° Reemplazo - Transformación estrella a triángulo}
	\label{1reemplazo} 
\end{figure}


Para el segundo reemplazo, se reagrupan las impedancias
 de la siguiente manera:
 \vspace{2mm}
\begin{align}
		Z_{A'} &= R_1 + Z_{A} \\
		Z_{B'} &= R_1 + Z_{B} \\
		Z_{C'} &= \frac{1}{sC_2} + Z_{C}
\end{align}
\vspace{2mm}
Se hacen las 
siguientes consideraciones:

\begin{align*}
	Z_{eq'} &= Z_{A'} + Z_{B'} + Z_{C'}
\end{align*}
\vspace{2mm}

Consecuentemente, se realiza la segunda transformación de Kenelly, pasando 
de un modelo estrella un triángulo. Obteniéndose las siguientes expresiones:

\begin{align}
	Z_{A'B'} &= \frac{Z_{eq'}}{Z_{C'}} \\
	Z_{B'C'} &= \frac{Z_{eq'}}{Z_{A'}}  \\
	Z_{C'A'} &= \frac{Z_{eq'}}{Z_{B'}} 
\end{align}
\vspace{2mm}


\begin{figure}[H]
	\centering
	\includegraphics[width=0.9\textwidth]{../Ejercicio4-EcualizadorDeFase/Informe/2cambioTriangulo.png}
	\caption{2° Reemplazo - Transformación estrella a triángulo}
	\label{2reemplazo} 
\end{figure}

Por último, simplificamos las impedancias que estaban en paralelo obteniendo
 un circuito de 3 impedancias, representado por la figura
  [\ref{Cir_Final}], mucho más simple de resolver.

 \begin{align}
	Z_{1} & = Z_{C'A'}//R_3 \\
	Z_{2} & = Z_{B'C'}//R_3 \\ 
	Z_{3} & = Z_{A'B'}//R_3
\end{align} 
\vspace{2mm}

\begin{figure}[h]
	\centering
	\includegraphics[width=0.45\textwidth]{../Ejercicio4-EcualizadorDeFase/Informe/EcFinal.png}
	\caption{Circuito simplificado}
	\label{Cir_Final}
\end{figure}

Debido al gran trabajo de cálculo requerido se decidió utilizar 
un programa matemático para asistirnos en el despeje de ecuaciones. 
El programa utilizado fue Matlab, del cual se desprenden las 
siguientes ecuaciones: \par 

\vspace{2mm}
\begin{align}
	Z_1 &= \frac{R_3(\alpha_1s^2+\alpha_2s+\alpha3)}{s^2(\alpha_1+C_1C_2R_1R_3R_{21}+C_1C_2R_1R_3R_{22})+s(\alpha_2+R_1R_3C_2+R_{22}R_3C_2)+\alpha_3} \label{z1} \\ 	
	Z_2 &= \frac{R_3(\alpha_1s^2+\alpha_2s+\alpha3)}{s^2(\alpha_1+C_1C_2R_1R_3R_{21}+C_1C_2R_1R_3R_{22})+s(\alpha_2+R_1R_3C_2+R_{21}R_3C_2)+\alpha_3} \label{z2} \\ 	
	Z_3 &= \frac{\alpha_1s^2+\alpha_2s+\alpha3}{s^2(C_1C_2R_{21}R_{22})+s(C_1R_{21}+C_1R_{22})+1} \label{z3}  	
\end{align}
\vspace{2mm}
Considerando las siguientes constantes:
\vspace{2mm}
\begin{align}
	\alpha_1 &= C_1C_2R_1^2R_{21}+C_1C_2R_1^2R_{22}+2C_1C_2R_1R_{21}R_{22}  \label{alpha1} \\ 	
	\alpha_2 &= 2C_1R_1R_{21}+2C_1R_1R_{22}+R_1^2C_2+R_1R_{21}C_2+R_1R_{22}C_2+R_{21}R_{22}C_2 \label{alpha2} \\ 	
	\alpha_3 &=  2R_1+R_{21}+R_{22}\label{alpha3}  	
\end{align}
\vspace{2mm}

Aplicando las expresiones obtenidas al principio de la seccción en 
[\ref{expresion_1}] y [\ref{expresion_2}], se pueden simplificar las ecuaciones 
[\ref{alpha1}],  [\ref{alpha2}] y [\ref{alpha3}] como:

\vspace{2mm}
\begin{align}
	\alpha_1 &= C_1C_2R_1^2R_2+2C_1C_2R_1R_2^2(L-L^2)  \label{alpha1_simplificado} \\ 	
	\alpha_2 &= 2C_1R_1R_2+R_1^2C_2+R_1R_2C_2+R_2^2(L-L^2)C_2 \label{alpha2_simplificado} \\ 	
	\alpha_3 &=  2R_1+R_2\label{alpha3_simplificado}  	
\end{align}
\vspace{2mm}

Por otro lado, utilizando las mismas expresiones se pueden simplificar las ecuaciones 
[\ref{z1}],  [\ref{z2}] y [\ref{z3}] como:

\vspace{2mm}
\begin{align}
	Z_1 &= \frac{R_3(\alpha_1s^2+\alpha_2s+\alpha3)}{s^2(\alpha_1+C_1C_2R_1R_3R_2)+s(\alpha_2+R_1R_3C_2+R_{22}R_3C_2)+\alpha_3} \label{z1_simplificado} \\ 	
	Z_2 &= \frac{R_3(\alpha_1s^2+\alpha_2s+\alpha3)}{s^2(\alpha_1+C_1C_2R_1R_3R_2)+s(\alpha_2+R_1R_3C_2+R_{21}R_3C_2)+\alpha_3} \label{z2_simplificado} \\ 	
	Z_3 &= \frac{\alpha_1s^2+\alpha_2s+\alpha3}{s^2(C_1C_2R_2^2(L-L^2))+s(C_1R_2)+1} \label{z3_simplificado}  	
\end{align}
\vspace{2mm}

A continuación, se realizarán dos análisis para obtener la función transferencia 
del circuito. La primera considerando un amplificador operacional ideal 
que implica las siguientes condiciones:

\begin{itemize}
	\item $A_{0}=\infty$
	\item $r_{in}=\infty$	
	\item $r_{o}=0$
\end{itemize}
De esta manera, se puede considerar una tierra virtual en la salida 
inversora del \textit{opamp}, obteniéndose la siguiente transferencia:

\begin{align}
	H_I(\$)=\frac{V_{out}}{V_{in}}=-\frac{Z_{2}}{Z_{1}}
	\label{trans_ideal}
\end{align}
 \vspace{2mm}

 Por otro lado, se analiza el caso de un \textit{opamp} no ideal, que 
 implica las siguientes condiciones:
 \begin{itemize}
	\item $A=\frac{A_{0}}{\left(1+\frac{s}{w_p}\right)}\neq\infty$
	\item $r_{in}\neq\infty$	
	\item $r_{o}\neq0$	
\end{itemize}
Este caso, al considerar menos aproximaciones, implica una mayor correlación 
con el funcionamiento empírico del circuito. Es necesario también 
utilizar las siguientes expresiones:

\begin{align}
	V_{out} & = A(V^+-V^-) \\
	V^+ & = 0 \\
	V^- & = V_{in} - I_1Z_1\\
	|I_1| & = \frac{V^-}{Z_{inp}} + \frac{V^--V_{out}}{Z_2}
\end{align} 
\vspace{2mm}

El circuito representado por dichas ecuaciones se puede apreciar 
en la figura [\ref{circuito_no_ideal}]. \par 
\begin{figure}[h]
	\centering
	\includegraphics[width=0.45\textwidth]{../Ejercicio4-EcualizadorDeFase/Informe/circuito_no_ideal.png}
	\caption{Circuito resultante sin aproximaciones}
	\label{circuito_no_ideal}
\end{figure}


Finalmente, de las ecuaciones presentadas se puede despejar la transferencia 
no ideal del sistema como:

\begin{align}
	H_{NI}(\$)=\frac{V_{out}}{V_{in}}=\frac{A_0w}{Z_{1}Ys+w(Z_1Y-A_0\frac{Z_1}{Z_2})}
	\label{trans_no_ideal}
\end{align}
 \vspace{2mm}
Donde $Y$ está dado por:
\begin{align}
	Y=\frac{1}{Z_{in}}+\frac{1}{Z_1}+\frac{1}{Z_2}
	\label{Y}
\end{align}
 \vspace{2mm}

 A partir de [\ref{trans_no_ideal}] es posible llegar a [\ref{trans_ideal}] considerando:

 \begin{align}
	\lim_{A_0\to\infty} H_{NI}(\$) = H_{I}(\$)
	\label{Y}
\end{align}
 \vspace{2mm}

 Por ende, se considera la transferencia ideal para seguir despejando.

\begin{align}
	H_I(\$)=-\frac{Z_{2}}{Z_{1}}=-\frac{s^2(\alpha_1+C_1C_2R_1R_3R_2)+s(\alpha_2+R_1R_3C_2+R_{22}R_3C_2)+\alpha_3}{s^2(\alpha_1+C_1C_2R_1R_3R_2)+s(\alpha_2+R_1R_3C_2+R_{21}R_3C_2)+\alpha_3}
	\label{trans_ideal_aplicada}
\end{align}

 Por otro lado, con el fin de simplificar la transferencia obtenida
 se aplican las siguientes condiciones de diseño sobre la 
 ecuación [\ref{trans_ideal_aplicada}].

 \begin{equation}
	 R_3 \gg R_1 \hspace{4mm} \wedge \hspace{4mm} R_3=10R_2 \hspace{4mm} \wedge  \hspace{4mm} C_1=10C_2
 \end{equation}

 \vspace{2mm}

 Obteniéndose:
 \vspace{2mm}
 \begin{align}
	H(\$)=-\frac{\alpha_{1'}s^2+\alpha_{2'}s+\alpha_3}{\alpha_{1'}s^2+\alpha_{2''}s+\alpha_3}
	\label{trans_ideal_aplicada}
\end{align}
\vspace{2mm}
Donde las constantes están representadas por:

\vspace{2mm}
\begin{align}
	\alpha_{1'} &  \approx 20C_2^2R_1R_2^2(L-L^2) + 100C_2^2R_1R_2^2 \label{alpha1_prima} \\ 	
	\alpha_{2'} & \approx 31C_2R_1R_2+ C_2R_2^2(10-9L-L^2) \label{alpha2_prima} \\ 	
	\alpha_{2''} & \approx 31C_2R_1R_2+ C_2R_2^2(11L-L^2) \label{alpha2_primaprima}  	
\end{align}
\vspace{2mm}

Considerando la ecuación [\ref{trans_ideal_aplicada}], busco su variante normalizada 
para hallar la frecuencia de corte. 

\vspace{2mm}
\begin{align}
   H(\$)=-\frac{s^2\left(\frac{\alpha_{1'}}{\alpha_{3}}\right)+s\left(\frac{\alpha_{2'}}{\alpha_{3}}\right)+1}{s^2\left(\frac{\alpha_{1'}}{\alpha_{3}}\right)+s\left(\frac{\alpha_{2''}}{\alpha_{3}}\right)+1}
   \label{trans_ideal_aplicada_normalizada}
\end{align}
\vspace{2mm}

De [\ref{trans_ideal_aplicada_normalizada}] se puede despejar la frecuencia de corte del circuito
dada por:

\vspace{2mm}
\begin{align}
		\omega_0 &= \sqrt{\frac{\alpha_{3}}{\alpha_{1'}}} \\
		f_0 &= \frac{1}{2\pi} \sqrt{\frac{\alpha_{3}}{\alpha_{1'}}}    \label{w_0} 
\end{align}
\vspace{2mm}
Reemplazando [\ref{alpha1_prima}] y [\ref{alpha3_simplificado}] en [\ref{w_0}], se obtiene:

\vspace{2mm}
\begin{align}
		f_0 &= \frac{1}{2\pi} . \sqrt{ \frac{2R_1 + R_2 }{ 20 C_2^2 R_1 R_2^2 (L (1-L)) + 100R_1 R_2^2C_2^2 } }
\end{align}
\vspace{2mm}


\begin{equation}
	f_0 = \frac{1}{2\pi C_2 R_2} \sqrt{ \frac{2 + \frac{R_2}{R_1} }{ 20L(1-L) + 10 } }	
	\label{f0sinsimplificar}
\end{equation}
\vspace{3mm}

Considerando $10 \gg 20L(1-L)$, se obtiene la expresión propuesta por la cátedra.

\vspace{2mm}
\begin{equation}
		f_0 \approx \frac{ \sqrt{ 2 + \frac{R_2}{R_1} } }{ 20\pi C_2 R_2 }
		\label{f0final}
\end{equation}
\vspace{2mm}

Para finalizar el análisis se puede tomar la frecuencia de corte hallada 
y reemplazarla en el módulo de la función transferencia. De esta manera, 
tomando los valores extremos para $L$ se pueden obtener 
 las cotas para dicha función, 
dadas por la ecuación [\ref{A0cotas}].

\vspace{2mm}
\begin{equation}
		\frac{3R_1}{3R_1+R_2} \leq  A_0 \leq\frac{3R_1+R_2}{3R_1}
		\label{A0cotas}
\end{equation}
\vspace{2mm}



	


