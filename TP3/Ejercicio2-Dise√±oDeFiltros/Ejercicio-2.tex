En la presente sección, se implementerán cuatro filtros de segundo orden según las siguientes especificaciones:

% Please add the following required packages to your document preamble:
% \usepackage[table,xcdraw]{xcolor}
% If you use beamer only pass "xcolor=table" option, i.e. \documentclass[xcolor=table]{beamer}
\begin{table}[]
    \centering
    \begin{tabular}{|c|c|c|c|}
    \hline
    \rowcolor[HTML]{C0C0C0} 
    Tipo de Filtro & $f_p[Hz]$ & $f_a[Hz]$ & $f_c[Hz]$ \\ \hline
    Low-Pass       & 5000   & 17500  & -      \\ \hline
    High-Pass      & 3500   & 1000   & -      \\ \hline
    Band-Pass      & -      & -      & 10000  \\ \hline
    Band-Rejection & -      & -      & 1000   \\ \hline
    \end{tabular}
    \end{table}


\subsection{Gyrto}


1. Diseñar una función transferencia que cumpla con las especificaciones.
3
2. Diseñar un circuito que implemente la función transferencia utilizando un Gyrator. Justificar adecuadamente la elección de todos sus componentes y redactar una introducción teórica al tema.
3. Determinar rangos de operación en zona lineal. Se espera adecuada profundidad en este análisis.
4. Contrastar el diseño del circuito con las simulaciones correspondientes.
5. Implementar el circuito y comprobar su funcionamiento con las mediciones correspondientes.
6. Analizar el comportamiento del sistema en altas frecuencias.
7. Diseñar un PCB que contenga todos los circuitos pedidos (en el mismo PCB). A su vez, puede utilizarse un
sólo IC en la implementación pedida.