\subsection{Circuito Derivador}



\subsection{Circuito Integrador}

\subsubsection{Introducción}

Se realizó el análisis de un circuito integrador ideal, utilizando en este caso tres componentes, una Resistencia $R$,
un capacitor $C$ y un amplificador operacional. 
Cabe destacar que se considera un integrador ideal ya que a diferencia del circuito RC analizado en el primer trabajo práctico de laboratorio,
éste funcionará como integrador para cualquier frecuencia y no solo a frecuencias altas. 

Los valores nominales utilizados para la experiencia fueron:

\begin{itemize}
	\item $R: 5.1K \Omega$ 
	\item $C: 20nF$
	\item $OPAMP: LM833$
\end{itemize}

\begin{figure}[H]
    \centering 
    \includegraphics [scale=0.5] {../Ejercicio3-CircuitoIntegradoresyDerivadores/Imagenes/diagrama-integrador.png} 
    \caption{Diagrama del circuito integrador ideal empleado}
    \label{fig:emptyPlotTool}
\end{figure}

A continuación se procederá a calcular teóricamente el valor de las funciones transferencias para los casos en 
donde el amplificador operacional tiene un comportamiento ideal, con $A_{vol}$ finito y $A_{vol}(w)$ con polo dominante.

\subsubsection{Análisis de la Transferencia del Circuito Integrador - OPAMP ideal}

Para obtener la función transferencia en este caso, $H(S) = \frac{V_{out} (S)}{V_{in} (S)}$, partiremos de las siguientes condiciones
iniciales para el amplificador operacional:

\begin{itemize}
	\item $A_{vol}: \infty$
	\item $Z_{in}: \infty$
	\item $Z_{out}: 0$
\end{itemize}

\begin{figure}[H]
    \centering 
    \includegraphics [scale=0.5] {../Ejercicio3-CircuitoIntegradoresyDerivadores/Imagenes/diagrama-integrador-corrientes.png} 
    \caption{Diagrama del circuito integrador ideal empleado}
    \label{fig:emptyPlotTool}
\end{figure}

Podemos observar a simple vista que:

\begin{itemize}
	\item $i1 = -i2$
	\item $i1 = \frac {V_{in}-V^{-}}{R} $
	\item $i2 = \frac {V_{out}-V^{-}}{X_c}$
	\item $V_{out} = A_{vol}(V^{+}-V^{-})$
\end{itemize}

Como ${A_{vol} \to \infty}$ y $V_{out}$ es finito, ${(V^{+}-V^{-}) \to 0}$ y como $V^{+}$ está conectado a tierra,
$(V^{-}$ representa tierra virtual, por lo cual su valor es de $0V$.

Entonces, redefiniendo las ecuaciones anteriores:

\begin{itemize}
	\item $i1 = \frac{V_{in}}{R} $
	\item $i2 = \frac {V_{out}}{X_c}$
\end{itemize}

Siendo entonces:

$$ \frac{V_{in}}{R} = - (\frac{V_{out}}{X_c}) \Longrightarrow \frac{V_{out}}{V_{in}} = -\frac{X_c}{R} = - \frac{1}{SRC}$$

$$ H(S) = - \frac{1}{SRC}$$

Claramente se puede apreciar que este circuito se comportará como un integrador, ya que si antitransformamos la función de transferencia
obtenida implicará que para obtener $v_{out}(t)$ habrá que integrar $v_{in}(t)$ en el dominio del tiempo.

En las siguientes figuras, se puede apreciar el Diagrama de Bode para este caso.

\begin{figure}[H]
    \centering 
    \includegraphics [scale=1] {../Ejercicio3-CircuitoIntegradoresyDerivadores/Imagenes/diagrama-bode-ideal-amplitud.png} 
    \caption{Diagrama de BODE de Amplitud para OPAMP ideal}
    \label{fig:emptyPlotTool}
\end{figure}

\begin{figure}[H]
    \centering 
    \includegraphics [scale=1] {../Ejercicio3-CircuitoIntegradoresyDerivadores/Imagenes/diagrama-bode-ideal-fase.png} 
    \caption{Diagrama de BODE de Fase para OPAMP ideal}
    \label{fig:emptyPlotTool}
\end{figure}

\subsubsection{Análisis de la Transferencia del Circuito Integrador - OPAMP con A finito}

A diferencia del caso anterior, aquí la diferencia en el cálculo de la función transferencia, $H(S) = \frac{V_{out} (S)}{V_{in} (S)}$,
entre el amplificador operaciones ideal y éste será:

\begin{itemize}
	\item $A_{vol}: finito$
\end{itemize}

Utilizando las mismas relaciones mencionadas en el apartado anterior, podemos observar ahora que:


$$V_{out}=-A_{vol}.V^{-} \Longrightarrow V^{-} = \frac{-V_{out}}{A_{vol}}$$ 


Por lo tanto:

\begin{itemize}
	\item $i1 = \frac {V_{in}-V^{-}}{R} =  \frac {V_{in} + \frac{V_{out}}{A_{vol}}}{R}$
	\item $i2 = \frac {V_{out}-V^{-}}{X_c} = \frac {V_{out} + \frac{V_{out}}{A_{vol}}}{X_c}$
\end{itemize}

Siendo entonces:

$$ \frac {V_{in} + \frac{V_{out}}{A_{vol}}}{R} = -(\frac {V_{out} + \frac{V_{out}}{A_{vol}}}{X_c})
\Longrightarrow \frac{V_{out}}{V_{in}} = \frac{1}{SCR(1+\frac{1}{A_{vol}}+\frac{1}{A_{vol}SRC})}$$

Finalmente:

$$H(S)= \frac{1}{SCR(1+\frac{1}{A_{vol}})+\frac{1}{A_{vol}}}$$

Es importante notar que siendo la ganancia para el caso ideal (GI) $- \frac{1}{SRC}$,  la funcion
transferencia se puede representar como $H(S) = GI. \frac{1}{SCR(1+\frac{1}{A_{vol}})+\frac{1}{A_{vol}}}$
Si $A_{vol}$ es lo suficientemente grande, tendremos una función transferencia ideal nuevamente.

\begin{figure}[H]
    \centering 
    \includegraphics [scale=1] {../Ejercicio3-CircuitoIntegradoresyDerivadores/Imagenes/diagrama-bode-cideal-amplitud.png} 
    \caption{Diagrama de BODE de Amplitud para OPAMP con $A_{vol}$ finito}
    \label{fig:emptyPlotTool}
\end{figure}

\begin{figure}[H]
    \centering 
    \includegraphics [scale=1] {../Ejercicio3-CircuitoIntegradoresyDerivadores/Imagenes/diagrama-bode-cideal-fase.png} 
    \caption{Diagrama de BODE de Fase para OPAMP con $A_{vol}$ finito}
    \label{fig:emptyPlotTool}
\end{figure}

\subsubsection{Análisis de la Transferencia del Circuito Integrador - OPAMP con $A_{vol}(w)$}

En este ultimo caso de analisis, $A_{vol}$ no es constante sino que es función de la frecuencia según:

$$A_{vol}=\frac{A_0}{1+\frac{S}{w_b}}$$

Por lo cual la expresion para la funcion transferencia calculada en el caso anterior, quedara denominada por:

$$H(S)= \frac{1}{SCR(1+\frac{1+\frac{1}{SCR}}{A_{vol}})}\Longrightarrow H(S)= \frac{1}{SCR(1+\frac{1+\frac{1}{SCR}}{\frac{A_0}{1+\frac{S}{w_b}}})}$$ 

Reacomodando algebraicamente:

$$H(S)=- \frac{1}{S^2\frac{CR}{A_oW_b}+SCR(1 + \frac{1}{A_o}+\frac{1}{W_bA_oCR}) + \frac{1}{A_0}}$$

Podemos observar que si $A_o$ es muy grande, nuevamente estaremos en el caso donde la ganancia que obtendremos será la ideal para este circuito.

\begin{figure}[H]
    \centering 
    \includegraphics [scale=1] {../Ejercicio3-CircuitoIntegradoresyDerivadores/Imagenes/diagrama-bode-noideal-amplitud.png} 
    \caption{Diagrama de BODE de Amplitud para OPAMP con $A_{vol}(w)$ finito}
    \label{fig:emptyPlotTool}
\end{figure}

\begin{figure}[H]
    \centering 
    \includegraphics [scale=1] {../Ejercicio3-CircuitoIntegradoresyDerivadores/Imagenes/diagrama-bode-noideal-fase.png} 
    \caption{Diagrama de BODE de Fase para OPAMP con $A_{vol}(w)$ }
    \label{fig:emptyPlotTool}
\end{figure}

Comparando los tres casos, podemos observar que en determinadas frecuencias el comportamiento es identico:

\begin{figure}[H]
    \centering 
    \includegraphics [scale=1] {../Ejercicio3-CircuitoIntegradoresyDerivadores/Imagenes/comparativo-magnitud.png} 
    \caption{Diagrama de BODE de Amplitud para OPAMP comparativo }
    \label{fig:emptyPlotTool}
\end{figure}

\begin{figure}[H]
    \centering 
    \includegraphics [scale=1] {../Ejercicio3-CircuitoIntegradoresyDerivadores/Imagenes/comparativo-fase.png} 
    \caption{Diagrama de BODE de Fase para OPAMP comparativo }
    \label{fig:emptyPlotTool}
\end{figure}

\subsubsection{Análisis de Impedancia de Entrada al Circuito Integrador}

Para poder calcular teoricamente, la impedancia de entrada, $Z_{in}$, se utilizó el teorema de Miller tal que:

$$ V_{out}=-A_{vol}.V^-$$

Como para este caso, $K=-A_{vol}$, el circuito integrador con el amplificador operacional, queda expresado como:

\begin{figure}[H]
    \centering 
    \includegraphics [scale=0.7] {../Ejercicio3-CircuitoIntegradoresyDerivadores/Imagenes/miller-integrador.png} 
    \caption{Diagrama de BODE de Fase para OPAMP comparativo }
    \label{fig:emptyPlotTool}
\end{figure}

Como nos interesa $Z_{in}=\frac{V_{in}}{i_1}$, es muy sencillo ver que a la entrada del amplificador operacional no inversora no ingresa corriente,
por lo cual utilizando la ley de tensiones de Kirchoff:

$$ V_{in} = i_1.R + i_1.\frac{X_c}{1+A_{vol}} \longrightarrow \frac{V_{in}}{i_1}= R + \frac{X_c}{1+A_{vol}} \longrightarrow Z_{in}=R+\frac{1}{SC(1+A_0)}$$

Dicha impedancia puede verse expresada en las siguientes figuras:

\begin{figure}[H]
    \centering 
    \includegraphics [scale=1] {../Ejercicio3-CircuitoIntegradoresyDerivadores/Imagenes/zin-magnitud.png} 
    \caption{Diagrama de BODE de Fase para OPAMP comparativo }
    \label{fig:emptyPlotTool}
\end{figure}

\begin{figure}[H]
    \centering 
    \includegraphics [scale=1] {../Ejercicio3-CircuitoIntegradoresyDerivadores/Imagenes/zin-fase.png} 
    \caption{Diagrama de BODE de Fase para OPAMP comparativo }
    \label{fig:emptyPlotTool}
\end{figure}

Se puede observar que a frecuencias notablemente bajas, la impedancia tiende a aumentar ya que $Z_{in}$ es inversamente proporcional al valor de la frecuencia.
A frecuencias del orden de los miliHz en adelante se puede observar como el efecto del capacitor se reduce, dejando unicamente la componente de la R. Por lo cual 
podemos afirmar que $Z_{in}$ es aproximadamente $R$ para esos casos, ademas del efecto que genera $A_0$.
Lo mismo se puede observar con la fase que tiende a 0, ya que la componente compleja que aporta la capacitancia se ve reducida conforme aumenta la frecuencia.